\documentclass[a4paper,12pt]{article}
\usepackage[utf8]{inputenc}
\usepackage[T1]{fontenc}
\usepackage{lmodern}
\usepackage{geometry}
\geometry{margin=1in}
\usepackage{enumitem}
\usepackage{hyperref}
\hypersetup{colorlinks=true, linkcolor=black, urlcolor=blue}
\usepackage[table]{xcolor}
\usepackage[dvipsnames]{xcolor}
\usepackage{graphicx}
\usepackage{float}
\usepackage{booktabs}
\usepackage{sectsty}
\sectionfont{\large\bfseries}
\subsectionfont{\normalsize\bfseries}
\usepackage{tocloft}
\renewcommand{\cftsecleader}{\cftdotfill{\cftdotsep}}
\usepackage{parskip}
\setlength{\parskip}{0.5em}
\setlength{\parindent}{0pt}
\usepackage{fancyhdr}

\begin{document}

\pagestyle{fancy}

\fancyhead[L]{Vulnerability Report}
\fancyhead[R]{May 28, 2025}

\begin{titlepage}
    \centering
    \vspace*{2cm}
    {\Huge\bfseries Vulnerability Report\par}
    \vspace{1cm}
    {\Large Object: Group15\par}
    {\Large Date: May 23, 2025\par}
    \vspace{0.5cm}
    {\large\bfseries CONFIDENTIAL INFORMATION\par}
    \vspace{2cm}
    {\large Conducted by: \\ Bernardo Walker Leichtweis \\ Enzzo Machado Silvino \\ Cassio Vieceli Filho\par}
\end{titlepage}

\tableofcontents
\clearpage

\section{Executive Summary}
Over a period of 2 days, a penetration test (\textit{Gray Box}) was conducted on the internal infrastructure of \textbf{Group15}, focusing on the assets \texttt{192.168.9.50} and \texttt{192.168.9.34}. The test, carried out by Bernardo W. Leichtweis, Enzzo M. Silvino, and Cassio V. Filho, used credentials of a user with minimal privileges (\texttt{cebolinha:c3b0l1nh4}) and followed the \textbf{PTES} methodology.

The primary objective was to identify exploitable vulnerabilities that pose a risk to network security. A total of \textbf{ten vulnerabilities} were found, classified as:
\begin{itemize}
    \item \textbf{3 High}: Weak password policy in the \textit{Web Application}, test user with improper access, and exposed database credentials.
    \item \textbf{3 Medium}: Lack of protection against brute force attacks on \textit{SSH} and the \textit{Web Application}, as well as user enumeration based on error messages.
    \item \textbf{4 Low}: Exposure of service versions.
\end{itemize}

Immediate remediation of the high-severity vulnerabilities is recommended, with a focus on strengthening authentication policies and permission management.

\section{Introduction}
This report presents the results of the security assessment on the hosts \texttt{192.168.9.50} and \texttt{192.168.9.34} of Group15. The test, conducted in \textit{Gray Box} mode with test credentials (\texttt{cebolinha:c3b0l1nh4}), aimed to identify vulnerabilities that compromise the confidentiality, integrity, or availability of the systems.

\subsection{Objective}
To identify vulnerabilities in the specified systems, provide practical recommendations to mitigate risks, and improve Group15's security posture.

\subsection{Methodology}
The assessment followed the \textbf{PTES} standard, complemented by the \textbf{OWASP Top 10}, and was structured in seven phases:
\begin{itemize}
    \item \textbf{Pre-Engagement Interactions}: Definition of scope and rules.
    \item \textbf{Intelligence Gathering}: Scanning with \texttt{nmap}.
    \item \textbf{Threat Modeling}: Analysis of services (\textit{SSH}, \textit{nginx}, \textit{MySQL}).
    \item \textbf{Vulnerability Analysis}: Manual and automated checks.
    \item \textbf{Exploitability}: Validation of vulnerabilities.
    \item \textbf{Post-Exploitation}: Impact analysis.
    \item \textbf{Reporting}: Detailed documentation.
\end{itemize}

Tools such as \texttt{nmap}, \texttt{Burp Suite}, \texttt{Hydra}, and \texttt{Nessus} were used, complemented by manual verifications.

\section{Test Overview}
\begin{table}[ht]
    \centering
    \begin{tabular}{lc}
        \toprule
        \rowcolor{gray!20} \textbf{Category} & \textbf{Quantity} \\
        \midrule
        Total Unique Vulnerabilities & 10 \\
        Critical & 0 \\
        High & 3 \\
        Medium & 3 \\
        Low & 4 \\
        Informational & 0 \\ \hline
        Zero-Day & 0 \\
        Easily Exploitable & 2 \\
        \bottomrule
    \end{tabular}
\end{table}

\section{Project Scope}
\begin{enumerate}
    \item \texttt{192.168.9.50}
    \item \texttt{192.168.9.34}
\end{enumerate}

\section{Enumeration}
Active services were identified using \texttt{nmap} (\texttt{nmap -sV -A -Pn -p- -T4 <host>}):
\begin{table}[ht]
    \centering
    \begin{tabular}{lll}
        \toprule
        \rowcolor{gray!20} \textbf{Host} & \textbf{Port/Service} & \textbf{Details} \\
        \midrule
        \texttt{192.168.9.50} 
            & 22/tcp: ssh & OpenSSH 9.6p1 Ubuntu 3ubuntu13.11 \\
            & 80/tcp: http & Apache httpd \\
            & 443/tcp: ssl/http & Apache httpd \\ \hline
        \texttt{192.168.9.34} 
            & 22/tcp: ssh       & OpenSSH 9.6p1 Ubuntu 3ubuntu13.11 \\
            & 25/tcp: smtp      & (standard SMTP service) \\
            & 80/tcp: http      & nginx 1.24.0 (Ubuntu) \\
            & 110/tcp: pop3     & Dovecot pop3d \\
            & 139/tcp: netbios-ssn & Samba smbd 4 \\
            & 143/tcp: imap     & Dovecot imapd \\
            & 445/tcp: netbios-ssn & Samba smbd 4 \\
            & 465/tcp: ssl/smtps? &  \\
            & 587/tcp: smtp     &  \\
            & 993/tcp: imaps?   &  \\
            & 995/tcp: pop3s?   &  \\
            & 3306/tcp: mysql   & MySQL 8.0.42 \\
            & 4190/tcp: sieve   & Dovecot Pigeonhole sieve 1.0 \\
            & 8080/tcp: http    & nginx \\
            & 8443/tcp: ssl/http & nginx \\
        \bottomrule
    \end{tabular}
\end{table}

\clearpage

\section{Vulnerabilities}

\subsection*{\color{Red}High}
\begin{itemize}
    \item \textbf{[NOT REMEDIATED] Weak Password Policy} \\
    Total affected assets: 1 -- Remediated: 0 -- Retested: 0 -- Not remediated: 1
    \item \textbf{[NOT REMEDIATED] Broken Access Control} \\
    Total affected assets: 1 -- Remediated: 0 -- Retested: 0 -- Not remediated: 1
    \item \textbf{[NOT REMEDIATED] Exposed Database Credentials} \\
    Total affected assets: 1 -- Remediated: 0 -- Retested: 0 -- Not remediated: 1
\end{itemize}

\subsection*{\color{Orange}Medium}
\begin{itemize}
    \item \textbf{[NOT REMEDIATED] User Enumeration} \\
    Total affected assets: 1 -- Remediated: 0 -- Retested: 0 -- Not remediated: 1
    \item \textbf{[NOT REMEDIATED] Lack of Brute Force Protection (SSH)} \\
    Total affected assets: 2 -- Remediated: 0 -- Retested: 0 -- Not remediated: 2
    \item \textbf{[NOT REMEDIATED] Lack of Brute Force Protection (WebApp)} \\
    Total affected assets: 1 -- Remediated: 0 -- Retested: 0 -- Not remediated: 1
\end{itemize}

\subsection*{\color{NavyBlue}Low}
\begin{itemize}
    \item \textbf{[NOT REMEDIATED] Server Discloses Software Version (SSH)} \\
    Total affected assets: 2 -- Remediated: 0 -- Retested: 0 -- Not remediated: 2
    \item \textbf{[NOT REMEDIATED] Server Discloses Software Version (HTTP - 80)} \\
    Total affected assets: 1 -- Remediated: 0 -- Retested: 0 -- Not remediated: 1
    \item \textbf{[NOT REMEDIATED] Server Discloses Software Version (Dovecot - 4190)} \\
    Total affected assets: 1 -- Remediated: 0 -- Retested: 0 -- Not remediated: 1
    \item \textbf{[NOT REMEDIATED] Server Discloses Software Version (MySQL)} \\
    Total affected assets: 1 -- Remediated: 0 -- Retested: 0 -- Not remediated: 1
\end{itemize}

\clearpage

\section{Identified Vulnerabilities}

\subsection{Weak Password Policy}
\textbf{Severity:} \textcolor{Red}{High} \\
\textbf{Affected Asset:} \texttt{192.168.9.50} \\
\textbf{CVSS v3.1:} 8.8 (AV:N/AC:L/PR:L/UI:N/S:U/C:H/I:H/A:H) \\
\textbf{Reference:} CWE-521: Weak Password Requirements

\textbf{Description:}  
The system allows the creation of accounts with extremely weak passwords, such as \texttt{123}, \texttt{password}, or \texttt{abc123}, without enforcing any minimum complexity criteria. The absence of robust validation during new user registration directly compromises the effectiveness of the authentication mechanism, making it susceptible to brute force attacks and password guessing attempts, especially against accounts with elevated privileges.

\textbf{Attack Scenario:}  
An attacker registers an account in the system using a weak password like \texttt{abc123}. After verifying that the system does not enforce password security criteria, the attacker can deploy automated brute force tools to compromise legitimate user accounts, including administrators. This allows them to:  
\begin{itemize}
    \item Gain unauthorized access to restricted system areas.  
    \item Escalate privileges by compromising administrative accounts.  
    \item Modify critical settings or exfiltrate sensitive information.  
\end{itemize}
The lack of adequate protection against this type of attack represents a severe authentication control failure.

\textbf{Recommendations:}  
\begin{itemize}
    \item \textbf{Implement password complexity policy:} Require passwords with at least 12 characters, including uppercase and lowercase letters, numbers, and special characters.  
    \item \textbf{Server-side validation:} Ensure validations are applied on the backend to prevent bypass via manipulated requests.  
    \item \textbf{Lockout after failed attempts:} Implement account lockout mechanisms or introduce CAPTCHA after multiple failed login attempts.  
    \item \textbf{Multi-factor authentication (MFA):} Adopt MFA as an additional security layer in the login process.  
    \item \textbf{Periodic audits:} Review existing passwords and enforce password resets for users with weak credentials.  
\end{itemize}

\textbf{Proof of Concept (PoC):}  
\begin{enumerate}
    \item Access the system's registration page (\texttt{/register.php}).  
    \item Fill out the form, using \texttt{123} as the password.  
    \item Complete the registration and successfully authenticate using weak credentials.
\end{enumerate}

\begin{figure}[H]
    \centering
    \includegraphics[width=0.9\textwidth]{week-password.png}
    \caption{Account created with a weak password accepted by the system without restrictions.}
\end{figure}

\clearpage

\subsection{Exposed Database Credentials}
\textbf{Severity:} \textcolor{Red}{High} \\
\textbf{Affected Asset:} \texttt{192.168.9.50} \\
\textbf{CVSS v3.1:} 8.7 (AV:L/AC:L/PR:L/UI:N/S:U/C:H/I:H/A:N) \\
\textbf{References:} CWE-798: Use of Hard-coded Credentials

\textbf{Description:}  
During the analysis of the \texttt{config.php} file in the production environment, database credentials were found embedded directly in the source code: \texttt{@Usuario:@MalditaSenha1234567@}. This practice violates fundamental security principles, such as the separation of code and sensitive data. The exposure of credentials in files accessible to multiple users or via repositories poses a critical risk of database compromise and, consequently, the entire system.

\textbf{Attack Scenario:}  
An attacker with access to the application's file structure—through a misconfigured local account, permission flaws, or server vulnerabilities—accesses the \texttt{config.php} file. Upon viewing its contents, they extract the credentials \texttt{@Usuario:@MalditaSenha1234567@} and establish a direct connection to the database. This allows them to:  
\begin{itemize}
    \item Access, modify, or delete sensitive information directly in the database.  
    \item Create backdoors or hidden users for persistent access.  
    \item Inject malicious data or compromise system integrity.  
\end{itemize}
This flaw can also be exploited if the code is inadvertently exposed (e.g., Git misconfiguration, file leakage via HTTP, etc.).

\textbf{Recommendations:}  
\begin{itemize}
    \item \textbf{Remove credentials from source code:} Ensure sensitive data is loaded from environment variables or external, non-versioned configuration files.  
    \item \textbf{Use secret managers:} Adopt tools like \texttt{Vault}, \texttt{AWS Secrets Manager}, or \texttt{Doppler} to securely store and access secrets.  
    \item \textbf{Rotate credentials:} Immediately change exposed credentials and implement periodic rotation for critical system passwords.  
    \item \textbf{Restrict read permissions:} Verify and correct configuration file permissions to ensure only necessary services have access.  
\end{itemize}

\textbf{Proof of Concept (PoC):}  
\begin{enumerate}
    \item Access the server and navigate to the \texttt{/var/www/html} directory.  
    \item View the contents of the \texttt{config.php} file using the command \texttt{cat config.php}.  
    \item Identify the following lines with explicit credentials:
    \begin{verbatim}
    $dbuser = "@Usuario";
    $dbpass = "@MalditaSenha1234567@";
    \end{verbatim}
    \item Use tools like \texttt{mysql -h 192.168.9.34 -u @Usuario -p} to connect to the database with the obtained credentials.  
\end{enumerate}

\begin{figure}[H]
    \centering
    \includegraphics[width=0.9\textwidth]{db1.png}
    \caption{Source code snippet with database credentials embedded in the \texttt{config.php} file.}
\end{figure}

\begin{figure}[H]
    \centering
    \includegraphics[width=0.9\textwidth]{db2.png}
    \caption{MySQL access using the extracted credentials.}
\end{figure}

\clearpage

\subsection{Broken Access Control}
\textbf{Severity:} \textcolor{Red}{High} \\
\textbf{Affected Asset:} \texttt{192.168.9.50} \\
\textbf{CVSS v3.1:} 7.8 (AV:L/AC:L/PR:L/UI:N/S:U/C:H/I:H/A:N) \\
\textbf{References:} OWASP A05:2021 - Broken Access Control, CWE-269: Improper Privilege Management

\textbf{Description:}  
The test user \texttt{cebolinha}, with minimal system privileges, has direct access to the \texttt{/home/simple-blog} and \texttt{/var/www/html} directories, where the website's source code files are stored. This incorrect permission configuration allows complete reading of the application's code, severely compromising security, as an attacker can identify vulnerabilities, embedded credentials, or sensitive business logic.

\textbf{Attack Scenario:}  
After authenticating as the \texttt{cebolinha} user via SSH or a local terminal, the attacker navigates to the mentioned directories and accesses the web application files. With access to the source code, the attacker can:  
\begin{itemize}
    \item Identify hardcoded credentials (e.g., database or API connections).  
    \item Analyze the application logic for vulnerabilities, such as SQL injection, XSS, or authentication bypass.  
    \item Reuse code snippets for reverse engineering or malicious cloning of the application.  
\end{itemize}
This type of exposure represents a critical risk, especially in production environments.

\textbf{Recommendations:}  
\begin{itemize}
    \item \textbf{Review file and directory permissions:} Ensure only authorized services and users (e.g., the web server user) have read access to application directories.  
    \item \textbf{User isolation:} Apply isolation controls (e.g., chroot, containers, SELinux) to prevent regular users from accessing sensitive system areas.  
\end{itemize}

\textbf{Proof of Concept (PoC):}  
\begin{enumerate}
    \item Access the system with the test user \texttt{cebolinha}.  
    \item Run \texttt{cd /var/www/html} and use the \texttt{ls -la} command to list files.  
    \item View source code files with commands like \texttt{cat index.php}.  
\end{enumerate}

\begin{figure}[H]
    \centering
    \includegraphics[width=0.9\textwidth]{brokenaccess1.png}
    \caption{User with minimal permissions accessing the web application's source code.}
\end{figure}

\clearpage

\subsection{Lack of Brute Force Protection}
\textbf{Severity:} \textcolor{Orange}{Medium} \\
\textbf{Affected Assets:} \texttt{192.168.1.50}, \texttt{192.168.1.34} \\
\textbf{CVSS v3.1:} 6.5 (AV:N/AC:L/PR:N/UI:N/S:U/C:L/I:L/A:N) \\
\textbf{References:} CWE-307: Improper Restriction of Excessive Authentication Attempts

\subsubsection{SSH (Port 22)}
\textbf{Description:}  
The \texttt{OpenSSH} services on hosts \texttt{192.168.9.50} and \texttt{192.168.9.34} lack mechanisms to protect against brute force attacks, allowing unlimited authentication attempts. The absence of controls, such as IP blocking or progressive delays, facilitates automated attacks that test user and password combinations until valid credentials are found.

\textbf{Attack Scenario:}  
A remote attacker identifies the \texttt{SSH} service on port \texttt{22} using \texttt{nmap}. With tools like \texttt{Hydra} or \texttt{Medusa}, they execute a brute force attack, testing common password lists against known accounts (e.g., \texttt{root}, \texttt{admin}, \texttt{cebolinha}). Without attempt limitations, the attacker can:  
\begin{itemize}
    \item Compromise accounts with weak passwords, gaining system access.  
    \item Escalate privileges if the compromised account has elevated permissions.  
    \item Establish network persistence by installing backdoors or extracting data.  
\end{itemize}
This risk is exacerbated in internet-exposed networks, where automated bots frequently scan \texttt{SSH} services.

\textbf{Recommendations:}  
\begin{itemize}
    \item \textbf{Implement \texttt{fail2ban}:} Configure \texttt{fail2ban} to block IPs after 5 failed login attempts within 10 minutes.  
    \item \textbf{Progressive delays:} Adjust \texttt{sshd\_config} to introduce delays after authentication failures (e.g., \texttt{LoginGraceTime 30}).  
    \item \textbf{Multi-factor authentication:} Enable MFA for \texttt{SSH} using \texttt{Google Authenticator} or certificate-based SSH keys.  
    \item \textbf{Strong passwords:} Enforce complex password policies (minimum 12 characters, diverse characters).  
    \item \textbf{Access restriction:} Limit \texttt{SSH} connections to trusted IPs via \texttt{iptables} or \texttt{tcpwrappers}.  
\end{itemize}

\textbf{Proof of Concept (PoC):}  
\begin{enumerate}
    \item Execute a brute force attack with:\newline \texttt{hydra -l cebolinha -P /usr/share/wordlists/rockyou.txt ssh://192.168.1.50}.  
    \item Observe that the system allows hundreds of attempts without blocking or delay.  
    \item Demonstration: Successful authentication with \texttt{cebolinha:c3b0l1nh4} after multiple attempts.  
\end{enumerate}

\clearpage

\begin{figure}[ht]
    \centering
    \includegraphics[width=0.9\textwidth]{hydra3.png}
    \caption{Successful brute force attack with Hydra on the SSH service.}
\end{figure}

\clearpage

\subsubsection{Web Application}
\textbf{Description:}  
The web application hosted at \texttt{https://192.168.9.50} does not implement any mitigation mechanisms against repeated login attempts, such as account lockout, progressive wait times (rate limiting), or CAPTCHA. This lack of control allows an attacker to perform brute force attacks without any limitations, testing a large number of user and password combinations until gaining unauthorized access.

\textbf{Attack Scenario:}  
An attacker scans the IP \texttt{192.168.9.50} and identifies a login interface accessible via a browser or automated tools. Using scripts or tools like \texttt{Burp Suite Intruder}, \texttt{Hydra}, or \texttt{ffuf}, the attacker sends hundreds or thousands of POST requests with different password combinations without being blocked or alerted. This scenario allows them to:  
\begin{itemize}
    \item Compromise regular and administrative user accounts.  
    \item Collect statistics on weak passwords used on the platform.  
    \item Escalate privileges and compromise the environment.  
\end{itemize}
The absence of basic authentication controls makes the system vulnerable to automated and silent attacks.

\textbf{Recommendations:}  
\begin{itemize}
    \item \textbf{Implement attempt limitation:} Adopt temporary or tiered account lockout policies after multiple failed login attempts.  
    \item \textbf{Introduce CAPTCHA:} Include mechanisms like reCAPTCHA in the authentication form to hinder automated attacks.  
    \item \textbf{Logging and monitoring:} Log all login attempts and configure alerts to detect suspicious patterns (e.g., brute force).  
    \item \textbf{Multi-factor authentication (MFA):} Adopt MFA as an additional barrier to authentication, even if the password is discovered.  
\end{itemize}

\textbf{Proof of Concept (PoC):}  
\begin{enumerate}
    \item Access the application at \texttt{https://192.168.9.50/login}.  
    \item Use \texttt{Burp Suite Intruder} or \texttt{Hydra} to automate login attempts:
    \begin{verbatim}
    hydra -l admin -P /usr/share/wordlists/rockyou.txt 192.168.9.50 https-post-form "/login.php:username=^USER^&password=^PASS^:F=Invalid"
    \end{verbatim}
    \item Observe that the system accepts numerous requests without any mitigation mechanism.  
    \item After multiple attempts, authentication with valid credentials is possible, demonstrating the attack's feasibility.  
\end{enumerate}

\begin{figure}[H]
    \centering
    \includegraphics[width=0.9\textwidth]{hydra4.png}
    \caption{Brute force attack execution using Hydra against the login form.}
\end{figure}

\clearpage

\subsection{User Enumeration}
\textbf{Severity:} \textcolor{Orange}{Medium} \\
\textbf{Affected Assets:} \texttt{192.168.1.50}, \texttt{192.168.1.34} \\
\textbf{CVSS v3.1:} 5.3 (AV:N/AC:L/PR:N/UI:N/S:U/C:N/I:N/A:N) \\
\textbf{References:} CWE-204: Observable Response Discrepancy

\textbf{Description:}  
During the analysis of the web application's authentication functionality, it was observed that the system responds with distinct messages depending on whether the provided username exists. When attempting to authenticate with a non-existent username, the application returns \texttt{"User not found"}, while providing a valid username with an incorrect password yields \texttt{"Invalid credentials"}. This behavior exposes the application to user enumeration, facilitating targeted attacks on valid accounts.

\textbf{Attack Scenario:}  
An attacker can automate requests to the login endpoint using tools like \texttt{Burp Suite}, \texttt{ffuf}, or \texttt{curl} to test multiple usernames. Based on the response messages, the attacker can distinguish which usernames are valid in the system. This enables them to:  
\begin{itemize}
    \item Perform brute force attacks only on valid accounts, increasing success rates.  
    \item Conduct highly targeted phishing attacks based on real usernames.  
    \item Infer internal information about the system's user structure.  
\end{itemize}
This exposure represents a breach of confidentiality in the authentication process.

\textbf{Recommendations:}  
\begin{itemize}
    \item \textbf{Unify error messages:} Always return a generic message like \texttt{"Invalid username or password"} regardless of the error (non-existent user or incorrect password).  
    \item \textbf{Implement random delays:} Introduce random response times to hinder attack automation.  
    \item \textbf{Monitor login attempts:} Log and alert administrators about multiple login failures associated with specific IPs or users.  
    \item \textbf{Use token-based authentication:} Mechanisms like \texttt{OAuth2} and \texttt{OpenID Connect} help reduce exposure of traditional authentication logic.  
\end{itemize}

\textbf{Proof of Concept (PoC):}  
\begin{enumerate}
    \item Access the application's login form at \texttt{http://192.168.9.50/login}.  
    \item Enter a non-existent username (e.g., \texttt{admin123}) and any password. Observe the \texttt{"User not found"} message.  
    \item Then, enter a real username (e.g., \texttt{usuarioteste123}) and an incorrect password. Observe the \texttt{"Invalid credentials"} message.  
\end{enumerate}

\begin{figure}[H]
    \centering
    \includegraphics[width=0.9\textwidth]{notfound.png}
    \caption{Message displayed when entering a non-existent user.}
\end{figure}

\begin{figure}[H]
    \centering
    \includegraphics[width=0.9\textwidth]{invalid.png}
    \caption{Message displayed when entering a valid user with an incorrect password.}
\end{figure}

\clearpage

\subsection{Software Version Exposure}
\textbf{Severity:} \textcolor{NavyBlue}{Low} \\
\textbf{Affected Assets:} \texttt{192.168.1.50}, \texttt{192.168.1.34} \\
\textbf{CVSS v3.1:} 3.7 (AV:N/AC:H/PR:N/UI:N/S:U/C:L/I:N/A:N) \\
\textbf{References:} CWE-200: Exposure of Sensitive Information to an Unauthorized Actor

\subsubsection{SSH (Port 22)}
\textbf{Description:}  
The \texttt{OpenSSH} services on hosts \texttt{192.168.1.50} and \texttt{192.168.1.34} reveal the exact version (\texttt{OpenSSH 9.6p1 Ubuntu 3ubuntu13.11}) in the connection banner. This information allows attackers to identify public vulnerabilities associated with the specific version, facilitating targeted attacks.

\textbf{Attack Scenario:}  
An attacker uses \texttt{nmap} or \texttt{netcat} to capture the \texttt{SSH} banner (\texttt{SSH-2.0-OpenSSH\_9.6p1 Ubuntu-3ubuntu13.11}). With the version identified, they search vulnerability databases (e.g., CVE, Exploit-DB) for known issues, such as buffer overflows or authentication flaws. If an exploitable vulnerability exists, the attacker can:  
\begin{itemize}
    \item Compromise the \texttt{SSH} service, gaining system access.  
    \item Execute remote code or escalate privileges.  
\end{itemize}
This risk is higher if the \texttt{OpenSSH} version is outdated.

\textbf{Recommendations:}  
\begin{itemize}
    \item \textbf{Hide banner:} Configure \texttt{DebianBanner no} in the \texttt{/etc/ssh/sshd\_config} file and restart the service (\texttt{systemctl restart sshd}).  
    \item \textbf{Regular updates:} Keep \texttt{OpenSSH} updated with the latest security patches.  
    \item \textbf{Monitoring:} Log \texttt{SSH} connection attempts to detect malicious scans.  
    \item \textbf{Firewall:} Restrict access to port \texttt{22} to trusted IPs.  
\end{itemize}

\textbf{Proof of Concept (PoC):}  
\begin{enumerate}
    \item Run \texttt{nc 192.168.9.50 22} to capture the \texttt{SSH} banner.  
    \item Observe the response: \texttt{SSH-2.0-OpenSSH\_9.6p1 Ubuntu-3ubuntu13.11}.  
    \item Search for vulnerabilities associated with the version in \texttt{Exploit-DB}.  
\end{enumerate}

\begin{figure}[ht]
    \centering
    \includegraphics[width=0.9\textwidth]{ssh-banner3.png}
    \caption{SSH banner revealing the OpenSSH version.}
\end{figure}

\subsubsection{HTTP (Port 80 - nginx)}
\textbf{Description:}  
The \texttt{nginx} server on port \texttt{80} of host \texttt{192.168.9.34} exposes its version (\texttt{nginx/1.24}) in the HTTP \texttt{Server} header. This information can be used by attackers to identify known vulnerabilities associated with the specific version.

\textbf{Attack Scenario:}  
An attacker uses \texttt{curl -I http://192.168.9.34} to capture the \texttt{Server: nginx/1.24.0} header. With the version identified, they query vulnerability databases (e.g., CVE) for public exploits, such as configuration flaws or denial-of-service vulnerabilities. The attacker can:  
\begin{itemize}
    \item Exploit a known flaw to compromise the web server.  
    \item Perform additional reconnaissance based on the identified version.  
\end{itemize}

\textbf{Recommendations:}  
\begin{itemize}
    \item \textbf{Hide version:} Add \texttt{server\_tokens off;} to the \texttt{/etc/nginx/nginx.conf} file and restart the service (\texttt{systemctl restart nginx}).  
    \item \textbf{Updates:} Keep \texttt{nginx} updated to the latest version.  
    \item \textbf{Secure headers:} Configure additional security headers, such as \texttt{X-Content-Type-Options: nosniff}.  
\end{itemize}

\textbf{Proof of Concept (PoC):}  
\begin{enumerate}
    \item Run \texttt{curl -I http://192.168.9.24}.  
    \item Observe the \texttt{Server: nginx/1.24.0} header.  
\end{enumerate}

\begin{figure}[ht]
    \centering
    \includegraphics[width=0.9\textwidth]{nginxv2.png}
    \caption{HTTP header exposing the nginx version.}
\end{figure}

\clearpage

\subsubsection{MySQL (Port 3306)}
\textbf{Description:}  
The \texttt{MySQL} service on port \texttt{3306} of host \texttt{192.168.9.34} reveals its version (\texttt{MySQL 8.0.42}) during the initial TCP handshake. This exposure allows attackers to identify specific vulnerabilities associated with the database version.

\textbf{Attack Scenario:}  
An attacker uses \texttt{nmap} with the \texttt{mysql-info} script to capture the \texttt{MySQL 8.0.42} version during the handshake. With this information, they search the \texttt{NVD} for related CVEs, such as authentication or privilege escalation flaws. The attacker can:  
\begin{itemize}
    \item Exploit a vulnerability to access the database without credentials.  
    \item Combine with the weak password policy to gain full control.  
\end{itemize}

\textbf{Recommendations:}  
\begin{itemize}
    \item \textbf{Hide version:} Configure \texttt{MySQL} to not reveal the version in the handshake (requires advanced tweaks or proxies).  
    \item \textbf{Access restriction:} Limit port \texttt{3306} to trusted IPs via \texttt{iptables}.  
    \item \textbf{Updates:} Keep \texttt{MySQL} updated to the latest version.  
    \item \textbf{Monitoring:} Log \texttt{MySQL} connection attempts in logs.  
\end{itemize}

\textbf{Proof of Concept (PoC):}  
\begin{enumerate}
    \item Run \texttt{nmap -sV --script=mysql-info -p 3306 192.168.9.34}.  
    \item Observe the response containing \texttt{MySQL 8.0.42}.  
\end{enumerate}

\begin{figure}[ht]
    \centering
    \includegraphics[width=0.9\textwidth]{mysqlv2.png}
    \caption{MySQL handshake exposing the service version.}
\end{figure}

\clearpage

\subsubsection{Dovecot (Port 4190)}
\textbf{Description:}  
The service available on port \texttt{4190/tcp} of host \texttt{192.168.9.34} publicly exposes the \texttt{Dovecot} service banner, including the \texttt{Pigeonhole Sieve 1.0} plugin version. This information was identified through a service scan, revealing sensitive technical details that attackers can exploit for application fingerprinting and to search for known vulnerabilities specific to this version.

\textbf{Attack Scenario:}  
By performing a scan with \texttt{nmap} or a manual connection with \texttt{telnet}, a remote attacker obtains the following response from the service on port 4190:
\begin{verbatim}
4190/tcp open  sieve   Dovecot Pigeonhole sieve 1.0
\end{verbatim}
Based on this information, the attacker can:
\begin{itemize}
    \item Search for specific vulnerabilities in the \texttt{Pigeonhole Sieve 1.0} version in databases like CVE Details, Exploit-DB, or Rapid7.  
    \item Plan targeted attacks, such as remote code execution, denial of service (DoS), or privilege escalation.  
    \item Prioritize vulnerable targets in a larger network based on the exposed service version.  
\end{itemize}
The exposure of precise version details facilitates automated attacks and is classified as a security misconfiguration.

\textbf{Recommendations:}  
\begin{itemize}
    \item \textbf{Hide version banners:} Adjust the Dovecot configuration to not display server or plugin version details, such as Pigeonhole.  
    \item \textbf{Restrict access to port 4190:} Limit access to authorized IPs only through firewall rules or access control lists.  
    \item \textbf{Service update:} Ensure Dovecot and its modules (e.g., Pigeonhole) are updated to the latest secure versions.  
    \item \textbf{Disable unnecessary services:} If the ManageSieve protocol is not in use, disable port 4190.  
\end{itemize}

\textbf{Proof of Concept (PoC):}  
\begin{enumerate}
    \item Execute the following command:
    \begin{verbatim}
    nmap -sV -p 4190 192.168.9.34
    \end{verbatim}
    \item Alternatively, connect with:
    \begin{verbatim}
    telnet 192.168.9.34 4190
    \end{verbatim}
    \item Observe the banner displayed on the connection, revealing the Dovecot and plugin version.
\end{enumerate}

\begin{figure}[H]
    \centering
    \includegraphics[width=0.9\textwidth]{dovecot2.png}
    \caption{\texttt{nmap} result displaying the Dovecot and Pigeonhole Sieve plugin version.}
\end{figure}

\begin{figure}[H]
    \centering
    \includegraphics[width=0.9\textwidth]{dovecot1.png}
    \caption{\texttt{telnet} connection demonstrating the full banner with exposed version.}
\end{figure}

\clearpage

\section{Conclusion}

The penetration test conducted on the assets \texttt{192.168.9.50} and \texttt{192.168.9.34} revealed critical vulnerabilities that directly impact the infrastructure's security. Notably, the weak password policy, exposure of hardcoded credentials in the source code, and access control failures allow users with minimal privileges to access sensitive files, such as the application's source code. Additionally, the web application lacks mechanisms to prevent brute force attacks and allows user enumeration, significantly expanding the attack surface.

Additional vulnerabilities, such as the exposure of service versions (Dovecot, MySQL, nginx, and SSH), while individually less impactful, contribute to detailed reconnaissance of the environment by potential attackers. The combination of these flaws indicates an urgent need to adopt basic security controls, strengthen secure development practices, and review permissions and configurations for network-exposed services.

\section{Appendices}

\subsection{General Definitions}
\begin{table}[ht]
    \centering
    \begin{tabular}{lp{8cm}}
        \toprule
        \rowcolor{gray!20} \textbf{Term} & \textbf{Description} \\
        \midrule
        Total Unique Vulnerabilities & Distinct vulnerabilities identified within the scope. \\ \hline
        Zero-Day Vulnerability & Flaw unknown to the vendor, exploitable before patches are available. \\ \hline
        Easily Exploitable Vulnerability & Flaws detectable by automated tools or with public exploits. \\ \hline
        Critical Vulnerability & High impact on confidentiality, integrity, or availability. \\ \hline
        High Vulnerability & Requires immediate attention due to potential impact. \\ \hline
        Medium Vulnerability & Less urgent but can cause serious issues. \\ \hline
        Low Vulnerability & Not imminent but should be mitigated in the long term. \\
        \bottomrule
    \end{tabular}
    \caption{Definitions of terms used in the report.}
\end{table}

\clearpage

\subsection{Severity Levels}
\begin{table}[ht]
    \centering
    \begin{tabular}{lp{10cm}}
        \toprule
        \rowcolor{gray!20} \textbf{Level} & \textbf{Description} \\
        \midrule
        \textcolor{BrickRed}{Critical} & CVSS 9.0–10.0: High probability and impact. \\
        \textcolor{Red}{High} & CVSS 7.0–8.9: Medium to high probability and impact. \\
        \textcolor{Orange}{Medium} & CVSS 4.0–6.9: Low to medium probability or impact. \\
        \textcolor{NavyBlue}{Low} & CVSS 0.1–3.9: Low probability and impact. \\
        \textcolor{Periwinkle}{Informational} & No direct impact but provides useful information. \\
        \bottomrule
    \end{tabular}
    \caption{Severity levels of vulnerabilities.}
\end{table}

\subsection{Vulnerability Mapping by Asset}

\textbf{Asset: \texttt{192.168.9.50}}  
\begin{itemize}
    \item \textcolor{Red}{High}: Weak password policy (registration form).  
    \item \textcolor{Red}{High}: Broken access control (access to source code in \texttt{/home/simple-blog} and \texttt{/var/www/html}).  
    \item \textcolor{Red}{High}: Hardcoded database credentials (\texttt{config.php}).  
    \item \textcolor{Orange}{Medium}: User enumeration (distinct authentication messages).  
    \item \textcolor{Orange}{Medium}: Lack of brute force protection (web application, OpenSSH).  
    \item \textcolor{NavyBlue}{Low}: Service version exposure (\texttt{OpenSSH}).  
\end{itemize}

\textbf{Asset: \texttt{192.168.9.34}}  
\begin{itemize}
    \item \textcolor{Orange}{Medium}: Lack of brute force protection (\texttt{SSH}).  
    \item \textcolor{NavyBlue}{Low}: Service version exposure (\texttt{OpenSSH}, \texttt{nginx}, \texttt{MySQL}, \texttt{Dovecot}).  
\end{itemize}

\subsection{Tools Used}
\begin{table}[ht]
    \centering
    \begin{tabular}{ll}
        \toprule
        \rowcolor{gray!20} \textbf{Purpose} & \textbf{Tool} \\
        \midrule
        Network mapping and scanning & Nmap \\
        SMTP server testing & Swaks \\
        Communication and port testing & Netcat \\
        Brute force attacks & Hydra \\
        Vulnerability analysis & Nessus \\
        Web vulnerability analysis & SkipFish, Nikto, Gobuster \\ & Burp Suite, XSStrike, SQLMap \\
        Exploitation and post-exploitation & Metasploit \\
        Privilege escalation & LinPEAS \\
        \bottomrule
    \end{tabular}
\end{table}

\end{document}