\documentclass[a4paper,12pt]{article}
\usepackage[utf8]{inputenc}
\usepackage[T1]{fontenc}
\usepackage{lmodern}
\usepackage{geometry}
\geometry{margin=1in}
\usepackage{enumitem}
\usepackage{hyperref}
\hypersetup{colorlinks=true, linkcolor=black, urlcolor=blue}
\usepackage[table]{xcolor}
\usepackage[dvipsnames]{xcolor}
\usepackage{graphicx}
\usepackage{float}
\usepackage{booktabs}
\usepackage{sectsty}
\sectionfont{\large\bfseries}
\subsectionfont{\normalsize\bfseries}
\usepackage{tocloft}
\renewcommand{\cftsecleader}{\cftdotfill{\cftdotsep}}
\usepackage{parskip}
\setlength{\parskip}{0.5em}
\setlength{\parindent}{0pt}
\usepackage{fancyhdr}

\begin{document}

\pagestyle{fancy}

\fancyhead[L]{Vulnerability Report}
\fancyhead[R]{May 28, 2025}

\begin{titlepage}
    \centering
    \vspace*{2cm}
    {\Huge\bfseries Vulnerability Report\par}
    \vspace{1cm}
    {\Large Object: Group4\par}
    {\Large Date: May 23, 2025\par}
    \vspace{0.5cm}
    {\large\bfseries CONFIDENTIAL INFORMATION\par}
    \vspace{2cm}
    {\large Conducted by: \\ Bernardo Walker Leichtweis \\ Enzzo Machado Silvino \\ Cassio Vieceli Filho\par}
\end{titlepage}

\tableofcontents
\clearpage

\section{Executive Summary}
Over a period of 2 days, a penetration test (\textit{Gray Box}) was conducted on the internal infrastructure of \textbf{Group4}, focusing on the assets \texttt{192.168.2.50} and \texttt{192.168.2.34}. The test, carried out by Bernardo W. Leichtweis, Enzzo M. Silvino, and Cassio V. Filho, used credentials of a user with minimal privileges (\texttt{cebolinha:c3b0l1nh4}) and followed the \textbf{PTES} methodology.

The primary objective was to identify exploitable vulnerabilities that pose a risk to network security. A total of \textbf{three vulnerabilities} were found, classified as:
\begin{itemize}
    \item \textbf{1 Critical}: Hard-coded credentials and weak administrator password (\textit{admin:admin}) expose the web application to unauthorized access.
    \item \textbf{2 Low}: Exposure of service versions.
\end{itemize}

Immediate remediation of the critical vulnerability is recommended, with a focus on strengthening authentication policies and permission management.

\section{Introduction}
This report presents the results of the security assessment on the hosts \texttt{192.168.2.50} and \texttt{192.168.2.34} of Group4. The test, conducted in \textit{Gray Box} mode with test credentials (\texttt{cebolinha:c3b0l1nh4}), aimed to identify vulnerabilities that compromise the confidentiality, integrity, or availability of the systems.

\subsection{Objective}
To identify vulnerabilities in the specified systems, provide practical recommendations to mitigate risks, and improve Group4's security posture.

\subsection{Methodology}
The assessment followed the \textbf{PTES} standard, complemented by the \textbf{OWASP Top 10}, and was structured in seven phases:
\begin{itemize}
    \item \textbf{Pre-Engagement Interactions}: Definition of scope and rules.
    \item \textbf{Intelligence Gathering}: Scanning with \texttt{nmap}.
    \item \textbf{Threat Modeling}: Analysis of services (\textit{SSH}, \textit{nginx}, \textit{MySQL}).
    \item \textbf{Vulnerability Analysis}: Manual and automated checks.
    \item \textbf{Exploitability}: Validation of vulnerabilities.
    \item \textbf{Post-Exploitation}: Impact analysis.
    \item \textbf{Reporting}: Detailed documentation.
\end{itemize}

Tools such as \texttt{nmap}, \texttt{Burp Suite}, \texttt{Hydra}, and \texttt{Nessus} were used, complemented by manual verifications.

\clearpage

\section{Test Overview}
\begin{table}[ht]
    \centering
    \begin{tabular}{lc}
        \toprule
        \rowcolor{gray!20} \textbf{Category} & \textbf{Quantity} \\
        \midrule
        Total Unique Vulnerabilities & 3 \\
        Critical & 1 \\
        High & 0 \\
        Medium & 0 \\
        Low & 2 \\
        Informational & 0 \\ \hline
        Zero-Day & 0 \\
        Easily Exploitable & 1 \\
        \bottomrule
    \end{tabular}
\end{table}

\section{Project Scope}
\begin{enumerate}
    \item \texttt{192.168.2.50}
    \item \texttt{192.168.2.34}
\end{enumerate}

\section{Enumeration}
Active services were identified using \texttt{nmap} (\texttt{nmap -sV -A -Pn -p- -T4 <host>}):
\begin{table}[ht]
    \centering
    \begin{tabular}{lll}
        \toprule
        \rowcolor{gray!20} \textbf{Host} & \textbf{Port/Service} & \textbf{Details} \\
        \midrule
        \texttt{192.168.2.50} 
            & 22/tcp: ssh & OpenSSH 9.6p1 Ubuntu 3ubuntu13.11 \\
            & 25/tcp: smtp & Postfix smtpd \\
            & 80/tcp: http & Apache httpd 2.4.58 \\
            & 143/tcp: imap & Dovecot imapd \\
            & 993/tcp: ssl/imap & Dovecot imapd \\
            & 10050/tcp: tcpwrapped & (Unidentified service) \\ \hline
        \texttt{192.168.2.34} 
            & 22/tcp: ssh & OpenSSH 9.6p1 Ubuntu 3ubuntu13.11 \\
            & 25/tcp: smtp & Postfix smtpd \\
            & 80/tcp: http & Apache httpd 2.4.58 ((Ubuntu)) \\
            & 139/tcp: netbios-ssn & Samba smbd 4 \\
            & 445/tcp: netbios-ssn & Samba smbd 4 \\
            & 3306/tcp: mysql & MySQL (unauthorized) \\
            & 3389/tcp: ms-wbt-server & Microsoft Terminal Service \\
            & 5900/tcp: vnc & VNC (protocol 3.8) \\
            & 8200/tcp: http & Golang net/http server \\
            & 9200/tcp: ssl/http & Amazon OpenSearch REST API (Basic auth) \\
        \bottomrule
    \end{tabular}
\end{table}

\clearpage

\section{Vulnerabilities}

\subsection*{\color{BrickRed}Critical}
\begin{itemize}
    \item \textbf{[NOT REMEDIATED] Hard-Coded Weak Password} \\
    Total affected assets: 1 -- Remediated: 0 -- Retested: 0 -- Not remediated: 1
\end{itemize}

\subsection*{\color{NavyBlue}Low}
\begin{itemize}
    \item \textbf{[NOT REMEDIATED] Server Discloses Software Version (SSH)} \\
    Total affected assets: 2 -- Remediated: 0 -- Retested: 0 -- Not remediated: 2
    \item \textbf{[NOT REMEDIATED] Server Discloses Software Version (HTTP - 80)} \\
    Total affected assets: 2 -- Remediated: 0 -- Retested: 0 -- Not remediated: 2
\end{itemize}

\clearpage

\section{Identified Vulnerabilities}

\subsection{Hard-Coded Weak Password}
\textbf{Severity:} \textcolor{BrickRed}{Critical} \\
\textbf{Affected Asset:} \texttt{192.168.2.50} \\
\textbf{CVSS v3.1:} 9.1 (AV:N/AC:L/PR:N/UI:N/S:U/C:H/I:H/A:N) \\
\textbf{Reference:} CWE-798 – Use of Hard-coded Credentials, CWE-521: Weak Password Requirements

\textbf{Description:}  
It was identified that the application has hard-coded administrator credentials directly in the source code, with an extremely weak password (\texttt{admin:admin}). This practice represents a serious vulnerability, as it exposes sensitive information and facilitates unauthorized access to the system, especially in environments where the source code is accessible to users or attackers.

\textbf{Attack Scenario:}  
An attacker with access to the server (through prior exploitation or misconfigured permissions) examines the application directory files and finds administrator credentials embedded in the code. With this, they can:
\begin{itemize}
    \item Gain direct access to the application’s administrative interface (if operational).
    \item Use the credentials in other interfaces or services reusing the same username/password pair.
    \item Modify or exfiltrate sensitive data by exploiting administrative privileges.
\end{itemize}
This type of flaw represents a breach of basic security best practices and a violation of the principles of confidentiality and credential segregation.

\textbf{Recommendations:}  
\begin{itemize}
    \item \textbf{Remove hard-coded credentials:} Never include usernames or passwords directly in the source code.
    \item \textbf{Use environment variables or vaults:} Store credentials securely outside the code.
    \item \textbf{Enforce strong password policy:} Replace weak passwords with robust combinations of at least 12 characters, including letters, numbers, and symbols.
    \item \textbf{Audit and control access to code:} Ensure only authorized users have access to the application directory.
    \item \textbf{Periodic password rotation:} Implement policies for password expiration and rotation.
\end{itemize}

\textbf{Proof of Concept (PoC):}  
\begin{enumerate}
    \item Access the server or open a session with read permissions in the application directory: \texttt{/var/www/html/php}.
    \item Locate the file with embedded credentials (\texttt{login.php}).
    \item The weak and hard-coded password can be used for administrative authentication (if the system becomes operational again).
\end{enumerate}

\begin{figure}[H]
    \centering
    \includegraphics[width=0.9\textwidth]{adminadmin.png}
    \caption{Source code snippet exposing hard-coded credentials with a weak password.}
\end{figure}

\clearpage

\subsection{Software Version Exposure}
\textbf{Severity:} \textcolor{NavyBlue}{Low} \\
\textbf{Affected Assets:} \texttt{192.168.2.50}, \texttt{192.168.2.34} \\
\textbf{CVSS v3.1:} 3.7 (AV:N/AC:H/PR:N/UI:N/S:U/C:L/I:N/A:N) \\
\textbf{References:} CWE-200: Exposure of Sensitive Information to an Unauthorized Actor

\subsubsection{SSH (Port 22)}
\textbf{Description:}  
The \texttt{OpenSSH} services on hosts \texttt{192.168.2.50} and \texttt{192.168.2.34} reveal the exact version (\texttt{OpenSSH 9.6p1 Ubuntu 3ubuntu13.11}) in the connection banner. This information allows attackers to identify public vulnerabilities associated with the specific version, facilitating targeted attacks.

\textbf{Attack Scenario:}  
An attacker uses \texttt{nmap} or \texttt{netcat} to capture the \texttt{SSH} banner (\texttt{SSH-2.0-OpenSSH\_9.6p1 Ubuntu-3ubuntu13.11}). With the version identified, they search vulnerability databases (e.g., CVE, Exploit-DB) for known issues, such as buffer overflows or authentication flaws. If an exploitable vulnerability exists, the attacker can:
\begin{itemize}
    \item Compromise the \texttt{SSH} service, gaining system access.
    \item Execute remote code or escalate privileges.
\end{itemize}
This risk is higher if the \texttt{OpenSSH} version is outdated.

\textbf{Recommendations:}  
\begin{itemize}
    \item \textbf{Hide banner:} Configure \texttt{DebianBanner no} in the \texttt{/etc/ssh/sshd\_config} file and restart the service (\texttt{systemctl restart sshd}).
    \item \textbf{Regular updates:} Keep \texttt{OpenSSH} updated with the latest security patches.
    \item \textbf{Monitoring:} Log \texttt{SSH} connection attempts to detect malicious scans.
    \item \textbf{Firewall:} Restrict access to port \texttt{22} to trusted IPs.
\end{itemize}

\textbf{Proof of Concept (PoC):}  
\begin{enumerate}
    \item Run \texttt{nc 192.168.2.50 22} to capture the \texttt{SSH} banner.
    \item Observe the response: \texttt{SSH-2.0-OpenSSH\_9.6p1 Ubuntu-3ubuntu13.11}.
    \item Search for vulnerabilities associated with the version in \texttt{Exploit-DB}.
\end{enumerate}

\begin{figure}[ht]
    \centering
    \includegraphics[width=0.9\textwidth]{SSH-BANNER4.png}
    \caption{SSH banner revealing the OpenSSH version.}
\end{figure}

\subsubsection{HTTP (Port 80 - Apache)}
\textbf{Description:}  
The \texttt{Apache} server on port \texttt{80} of hosts \texttt{192.168.2.50} and \texttt{192.168.2.34} exposes its version (\texttt{Apache/2.4.58}) in the HTTP \texttt{Server} header. This information can be used by attackers to identify known vulnerabilities associated with the specific version.

\textbf{Attack Scenario:}  
An attacker uses \texttt{curl -I http://192.168.2.50} to capture the \texttt{Server: Apache/2.4.58} header. With the version identified, they query vulnerability databases (e.g., CVE) for public exploits, such as configuration flaws or denial-of-service vulnerabilities. The attacker can:
\begin{itemize}
    \item Exploit a known flaw to compromise the web server.
    \item Perform additional reconnaissance based on the identified version.
\end{itemize}

\textbf{Recommendations:}  
\begin{itemize}
    \item \textbf{Hide version:} Configure the \texttt{ServerTokens Prod} and \texttt{ServerSignature Off} directives in the \texttt{/etc/apache2/apache2.conf} file or equivalent, and restart the service with \texttt{systemctl restart apache2}.
    \item \textbf{Updates:} Keep \texttt{Apache httpd} updated to the latest stable version available in the distribution’s repository.
    \item \textbf{Secure headers:} Add security headers such as \texttt{X-Content-Type-Options "nosniff"}, \texttt{X-Frame-Options "DENY"}, and \texttt{Content-Security-Policy} using the \texttt{mod\_headers} module.
\end{itemize}

\textbf{Proof of Concept (PoC):}  
\begin{enumerate}
    \item Run \texttt{curl -I http://192.168.2.50}.
    \item Observe the \texttt{Server: Apache/2.4.58} header.
\end{enumerate}

\begin{figure}[ht]
    \centering
    \includegraphics[width=0.9\textwidth]{apache3.png}
    \caption{HTTP header exposing the Apache version.}
\end{figure}

\clearpage

\section{Conclusion}

The penetration test conducted on the assets \texttt{192.168.2.50} and \texttt{192.168.2.34} revealed critical vulnerabilities that compromise the security of the infrastructure. Notably, the use of hard-coded credentials with a weak administrative password represents a severe protection failure. Additionally, the exposure of SSH and Apache service versions was identified, facilitating environmental reconnaissance by potential attackers and increasing the attack surface.

These flaws highlight the urgent need to implement strict controls for credential management, concealment of sensitive information, and continuous service updates to mitigate risks and strengthen the network’s security posture.

\section{Appendices}

\subsection{General Definitions}
\begin{table}[ht]
    \centering
    \begin{tabular}{lp{8cm}}
        \toprule
        \rowcolor{gray!20} \textbf{Term} & \textbf{Description} \\
        \midrule
        Total Unique Vulnerabilities & Distinct vulnerabilities identified within the scope. \\ \hline
        Zero-Day Vulnerability & Flaw unknown to the vendor, exploitable before patches are available. \\ \hline
        Easily Exploitable Vulnerability & Flaws detectable by automated tools or with public exploits. \\ \hline
        Critical Vulnerability & High impact on confidentiality, integrity, or availability. \\ \hline
        High Vulnerability & Requires immediate attention due to potential impact. \\ \hline
        Medium Vulnerability & Less urgent but can cause serious issues. \\ \hline
        Low Vulnerability & Not imminent but should be mitigated in the long term. \\
        \bottomrule
    \end{tabular}
    \caption{Definitions of terms used in the report.}
\end{table}

\clearpage

\subsection{Severity Levels}
\begin{table}[ht]
    \centering
    \begin{tabular}{lp{10cm}}
        \toprule
        \rowcolor{gray!20} \textbf{Level} & \textbf{Description} \\
        \midrule
        \textcolor{BrickRed}{Critical} & CVSS 9.0–10.0: High probability and impact. \\
        \textcolor{Red}{High} & CVSS 7.0–8.9: Medium to high probability and impact. \\
        \textcolor{Orange}{Medium} & CVSS 4.0–6.9: Low to medium probability or impact. \\
        \textcolor{NavyBlue}{Low} & CVSS 0.1–3.9: Low probability and impact. \\
        \textcolor{Periwinkle}{Informational} & No direct impact but provides useful information. \\
        \bottomrule
    \end{tabular}
    \caption{Severity levels of vulnerabilities.}
\end{table}

\subsection{Vulnerability Mapping by Asset}

\textbf{Asset: \texttt{192.168.2.50}}  
\begin{itemize}
    \item \textcolor{BrickRed}{Critical}: Hard-Coded Weak Password.
    \item \textcolor{NavyBlue}{Low}: Service version exposure (\texttt{OpenSSH}, \texttt{Apache}).
\end{itemize}

\textbf{Asset: \texttt{192.168.2.34}}  
\begin{itemize}
    \item \textcolor{NavyBlue}{Low}: Service version exposure (\texttt{OpenSSH}, \texttt{Apache}).
\end{itemize}

\subsection{Tools Used}
\begin{table}[ht]
    \centering
    \begin{tabular}{ll}
        \toprule
        \rowcolor{gray!20} \textbf{Purpose} & \textbf{Tool} \\
        \midrule
        Network mapping and scanning & Nmap \\
        SMTP server testing & Swaks \\
        Communication and port testing & Netcat \\
        Brute force attacks & Hydra \\
        Vulnerability analysis & Nessus \\
        Web vulnerability analysis & SkipFish, Nikto, Gobuster \\ & Burp Suite, XSStrike, SQLMap \\
        Exploitation and post-exploitation & Metasploit \\
        Privilege escalation & LinPEAS \\
        \bottomrule
    \end{tabular}
\end{table}

\end{document}